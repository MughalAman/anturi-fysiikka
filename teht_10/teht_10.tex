\documentclass{report}

\input{../preamble}
\input{../macros}
\input{../letterfonts}

\title{\Huge{Anturifysiikka} Tehtävät 10}
\author{\huge{Aman Mughal}}
\date{25/04/2023}

\begin{document}

\maketitle

\qs{}{
Erään analogisen lämpötila-anturin mittausalue on -40℃ … +125℃.
Anturille suoritetaan kaksipistekalibrointi, jonka yhteydessä ulostulojännitteen
arvo mitataan kahdella eri lämpötilalla: t1 = -20℃ ja t2 = +100℃.
Kalibrointitulokset ovat U(t1) = 300 mV ja U(t2) = 1500 mV.

a) Määritä anturin siirtofunktio (ts. funktio, joka kertoo ulostulojännitteen
mitattavan suureen funktiona), kun se oletetaan lineaariseksi.
b) Mikä on anturin herkkyys?
c) Jos anturin ulostulo on 1000 mV, mikä on mitattavan lämpötilan arvo?
}

\pf{Vastaukset}{
    \begin{flalign*}
        & \text{(a)} \\
        & U(t_1) = at_1 + b \\
        & U(t_2) = at_2 + b \\\\
        & \text{Ratkaistaan yhtälöparista } a \text{ ja } b: \\
        & a = \frac{U(t_2) - U(t_1)}{t_2 - t_1} \\
        & b = U(t_1) - at_1 \\\\
        & \text{Saadaan: } \\
        & a = \frac{1500\text{ mV} - 300\text{ mV}}{100^\circ\text{C} - (-20)^\circ\text{C}} = \frac{1200\text{ mV}}{120^\circ\text{C}} = 10\text{ mV/}\degree\text{C} \\
        & b = 300\text{ mV} - 10\text{ mV/}\degree\text{C} \cdot (-20^\circ\text{C}) = 300\text{ mV} - (-200\text{ mV}) = 500\text{ mV} \\\\
        & \text{Siirtofunktio on siis: } \\
        & U(T) = aT + b = 10\text{ mV/}\degree\text{C} \cdot T + 500\text{ mV} \\\\
        & \text{(b) Anturin herkkyys on siirtofunktion kulmakerroin eli } 10\text{ mV/}\degree\text{C}. \\\\
        & \text{(c) Saadaan ratkaisemalla yhtälö } U(T) = 1000\text{ mV} \text{ siirtofunktiosta:} \\
        & T = \frac{1000\text{ mV} - 500\text{ mV}}{10\text{ mV/}\degree\text{C}} = 50^\circ\text{C}
        \end{flalign*}
}

\end{document}
