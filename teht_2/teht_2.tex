\documentclass{report}

\input{../preamble}
\input{../macros}
\input{../letterfonts}

\title{\Huge{Anturifysiikka} Tehtävät 2}
\author{\huge{Aman Mughal}}
\date{25/01/2023}

\begin{document}

\maketitle

\qs{}{
    Elektroni liikkuu nopeudella $1,9 \cdot 10^7$ m/s tasaiseen magneettikenttään, jonka
    magneettivuon tiheys on 48 µT ja suunta kohtisuorassa elektronin nopeutta
    vastaan. Kuinka suuri on elektronin radan säde magneettikentässä? [ 2,3 m ]
}

\pf{Vastaus}{
    \begin{align*}
        r &= m * v / (q * B)  \\\\
        r &= \frac{9,1 \cdot 10^{-31} \text{ kg} \cdot 1,9 \cdot 10^7 \text{ m/s}}{1,6 \cdot 10^{-19} \text{ C} \cdot 48 \cdot 10^{-6} \text{ T}} \\\\
        r &= 2,3 \text{ m}
    \end{align*}
}

\qs{}{
    Kuinka suurella jännite-erolla edellisen tehtävän elektronit on kiihdytetty
    annetun nopeuden saavuttamiseksi? Voit olettaa, että elektronit ovat lähteneet
    liikkeelle levosta. [ 1,0 kV ]
}

\pf{Vastaus}{
    \begin{align*}
        a &= \frac{(1,9 \cdot 10^7)^2}{2.3} = 3.04 \cdot 10^{13}  \text{ m}/s^2\\\\
        t &= \frac{2 \pi r}{v} \\\\
        t &= \frac{2 \pi \cdot 2,3 \text{ m}}{1,9 \cdot 10^7 \text{ m/s}} \\\\
        t &= 0,74 \cdot 10^{-6} \text{ s} = 0,74 \mu \text{s} \\\\
        d &= \frac{1}{2} \cdot a \cdot t^2 \\\\
        d &= \frac{1}{2} \cdot (3.04 \cdot 10^{13} \text{ m}/s^2) \cdot (0,74 \cdot 10^{-6} \text{ s})^2\\\\
        \Delta V &= W / q = (q/m) \cdot a \cdot d \\\\
        \Delta V &= (4.307 \cdot 10^{-28} C/kg) \cdot (3.04 \cdot 10^{13} m/s^2) \cdot (8,32 \text{ m}) \\\\
        \Delta V &= 1,0  \cdot 10^6 \text{ V} = 1,0 kV
    \end{align*}
}

\newpage

\qs{}{
    Kuinka pitkä aika tehtävän 1 elektroneilla kuluu yhteen ratakierrokseen?
    Millaista taajuutta kyseinen jaksollinen liike vastaa? [ 0,74 µs; 1,3 MHz ]
}

\pf{Vastaus}{
    \begin{align*}
        T &= \frac{2 \pi r}{v} \\\\
        T &= \frac{2 \pi \cdot 2,3 \text{ m}}{1,9 \cdot 10^7 \text{ m/s}} \\\\
        T &= 0,74 \cdot 10^{-6} \text{ s} = 0,74 \mu \text{s} \\\\
        f &= \frac{1}{T} \\\\
        f &= \frac{1}{0,74 \cdot 10^{-6} \text{ s}} \\\\
        f &= 1,3 \cdot 10^6 \text{ Hz} = \text{1,3 MHz}
    \end{align*}
}

\newpage


\qs{}{
    220 km/s:n vauhtiin kiihdytetyt protonit
    ohjataan poikittaissuuntaiseen homogeeniseen
    magneettikenttään, jonka vaikutuksesta niiden
    nopeuden suunta kääntyy päinvastaiseksi.
    Oheisen kuvan esittämä tulo- ja lähtösuuntien
    välinen poikittaissiirtymä d on 40 cm. a) Määritä
    tämän perusteella (kuvassa harmaan alueen)
    magneettivuon tiheyden suuruus, ja b) päättele
    sen suunta. c) Kuinka kauan protoni viettää aikaa
    magneettikentässä? [a) 11 mT c) 2,9 µs ]
}

\pf{a) Vastaus}{
    \begin{align*}
        r &= \frac{40 \text{ cm}}{2}  = 20 \text{ cm} \\\\
        B &= \frac{mv}{q \cdot r} \\\\
        B &= \frac{1,672 \cdot 10^{-27} \text{ kg} \cdot 220 \cdot 10^3 \text{ m/s}}{1,6 \cdot 10^{-19} \text{ C} \cdot 20 \cdot 10^{-2} \text{ m}} \\\\
        B &= 11 \cdot 10^{-3} \text{ T} = 11 \text{ mT}
    \end{align*}
}

\pf{c) Vastaus}{
    \begin{align*}
        \frac{T}{2} &= \pi \cdot \frac{m}{q} \cdot B \\\\
        \frac{T}{2} &= \pi \cdot \frac{1,672 \cdot 10^{-27} \text{ kg}}{1,6 \cdot 10^{-19} \text{ C}} \cdot 11 \cdot 10^{-3} \text{ T} \\\\
        \frac{T}{2} &= 2,9 \cdot 10^{-6} \text{ s} = 2,9 \mu \text{s} 
    \end{align*}
}


\end{document}
