\documentclass{report}

\input{../preamble}
\input{../macros}
\input{../letterfonts}

\title{\Huge{Anturifysiikka} Tehtävät 1}
\author{\huge{Aman Mughal}}
\date{18/01/2023}

\begin{document}

\maketitle

\qs{}{
Magneettivuon tiheys 7,4 cm:n etäisyydellä pitkästä ja suorasta johtimesta
on 96 µT. Millä etäisyydellä johtimesta magneettivuon tiheys on 20 µT? [ 36 cm ]
}

\pf{Vastaus}{
    \begin{align*}
        \frac{B_1}{B_2} & = \frac{r_2}{r_1} \\\\
        \Rightarrow B_2 & = B_1 \cdot \frac{r_1}{r_2} \\\\
        20  \mu \text{T} &= 96 \mu \text{T} \cdot \frac{7.4 \text{ cm}}{r2} \\\\
        %solve for x
        \Rightarrow r_2 &= \frac{96 \mu \text{T} \cdot 7.4 \text{ cm}}{20 \mu \text{T}}\\\\
        r_2 &= \frac{710.4 \text{ cm}}{20} \approx 36 \text{ cm}
    \end{align*}
}

\qs{}{
    Kaksi yhdensuuntaista pitkää ja suoraa virtajohdinta on 9,7 cm:n
    etäisyydellä toisistaan. Määritä magneettivuon tiheys johtimien yhdysjanan
    keskipisteessä, kun johtimissa kulkevat 53 A:n ja 82 A:n virrat a) samaan
    suuntaan b) vastakkaisiin suuntiin. [ a) 0,12 mT b) 0,56 mT ]
}

\pf{a) Vastaus}{
    %calculate the magnetic field at the center of the two wires
    \begin{align*}
        B & = \frac{\mu_0 I}{2 \pi r} \\\\
        %plug in the values
        B & = \frac{4 \pi \cdot 10^{-7} \cdot 53 A}{2 \pi \cdot 9.7 \cdot 10^{-2}} \\\\
        %calculate the answer
        B & = 0.12 \cdot 10^{-3} \text{ T} = 0.12 \text{ mT}
    \end{align*}
}

\pf{b) Vastaus}{
    %calculate the magnetic field at the center of the two wires
    \begin{align*}
        B & = \frac{\mu_0 I}{2 \pi r}                                              \\\\
        %plug in the values
        B & = \frac{4 \pi \cdot 10^{-7} \cdot (82 A - 53 A)}{2 \pi \cdot 9.7 \cdot 10^{-3}}\\\\
        %calculate the answer
        B & = 0.56 \cdot 10^{-3} \text{ T} = 0.56 \text{ mT}
    \end{align*}
}

\qs{}{
    Ilmatäytteisen solenoidin pituus on 40 cm ja ympyrän muotoisen
    poikkileikkauksen halkaisija 28 mm. Johtimessa kulkevan virran suurin sallittu
    arvo on 12 A. Magneettivuon tiheydeksi solenoidin keskiosassa halutaan 35
    mT. a) Kuinka monta johdinkierrosta pituusyksikköä kohti on solenoidissa
    vähintään oltava? b) Määritä johtimen pituus. [ a) 2300 kierrosta/m b) 82 m ]
}

\pf{a) Vastaus}{
    %calculate the number of turns per meter
    \begin{align*}
        B &= \mu n I 
        \Rightarrow n = \frac{B}{\mu I} \\\\
        n &= \frac{35 \cdot 10^{-3}}{4 \pi \cdot 10^{-7} \cdot 12 A} \approx 2300 \text{ kierrosta/m} \\\\
    \end{align*}
}

\pf{b) Vastaus}{
    %calculate the length of the wire
    \begin{align*}
        B &= \frac{\mu_0 n I}{l} 
        \Rightarrow n = \frac{B \cdot l}{\mu_0 I} \\\\
        l_w &= \pi d \cdot \frac{B \cdot l}{\mu_0 I} \\\\
        l_w &= \pi \cdot 28 \text{ mm} \cdot 10^{-3} \cdot \frac{35 \cdot 10^{-3} \cdot 40 \cdot 10^{-2} \text{ cm}}{4 \pi \cdot 10^{-7} \cdot 12 A} \\\\
        l_w &\approx 82 \text{ m}
    \end{align*}
}

\qs{}{
Elektroni liikkuu nopeudella $1,9 \cdot 10^7$
m/s tasaiseen magneettikenttään,
jonka magneettivuon tiheys on 48 µT ja suunta kohtisuorassa elektronin
nopeutta vastaan. Kuinka suuri magneettinen voima elektroniin vaikuttaa? $ 1,5 \cdot 10^{-16} N $
}

\pf{Vastaus}{
    %calculate the magnetic force on the electron the result
    \begin{align*}
        F & = q \cdot v \cdot B                                                                              \\\\
        %plug in the values
        F & = 1.6 \cdot 10^{-19} \text{ C} \cdot 1.9 \cdot 10^7 \text{ m/s} \cdot 48 \cdot 10^{-6} \text{ T} \\\\
        %calculate the answer
        F & \approx 1.5 \cdot 10^{-16} \text{ N}
    \end{align*}
}

\end{document}
