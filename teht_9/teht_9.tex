\documentclass{report}

\input{../preamble}
\input{../macros}
\input{../letterfonts}

\title{\Huge{Anturifysiikka} Tehtävät 9}
\author{\huge{Aman Mughal}}
\date{10/04/2023}

\begin{document}

\maketitle

\qs{}{
Ilmatäytteisen toroidikäämin säde on 36 cm ja poikkileikkauksen pinta-ala
5,9 $cm^2$
. Kuinka suuri täytyy kierrosluvun olla, jotta käämin induktanssi olisi 50
µH? [ 390 ]
}

\pf{Vastaus}{
    \begin{align*}
        N &= \sqrt{\frac{2 \pi \cdot r \cdot L}{\mu \cdot A}} \\
        r &= 0.36 \text{ m} \\
        A &= 5.9 \cdot 10^{-4} \text{ m}^2 \\
        N &= \sqrt{\frac{2 \pi \cdot 0.36 m \cdot 50 \cdot 10^{-6} H}{4 \pi \cdot 10^{-7} \frac{H}{m} \cdot 5.9 \cdot 10^{-4} m^2}} \\
        N &\approx \underline{390}
    \end{align*}
}

\newpage

\qs{}{
Käämissä kulkeva sähkövirta pienenee 2,1 sekunnissa tasaisesti arvosta 2,2
A arvoon 1,6 A, pysyy vakiona 3,0 sekuntia ja kasvaa 4,4 sekunnissa tasaisesti
arvoon 3,6 A. Käämin induktanssi on 34 mH. Piirrä kuvaaja, joka esittää käämin
lähdejännitettä ajan funktiona.
[ 0 … 2,1 s: 9,7 mV; 2,1 s … 5,1 s: 0 V; 5,1 s … 9,5 s: -15,5 mV ]
}

\pf{Vastaus}{

}

\newpage

\qs{}{
Käämin läpi navasta A napaan B kulkee muuttuva tasavirta. Kun virta on 2,5
A, potentiaalieroksi $V_B - V_A$ mitataan $-33 V$. Käämin resistanssi on 12 $\Omega$ ja
induktanssi 0,50 H. a) Onko virta kasvamassa vai pienenemässä? b) Määritä
virran muutosnopeus. [ a) kasvamassa b) 6,0 A/s ]
}

\pf{Vastaus}{
    \begin{align*}
        -\Delta V &= L(\Delta l/\Delta t) + IR \\
        -(\ -33 V)\ &= 0,50 H \cdot \frac{\Delta l}{\Delta t} + 2.5 A \cdot 12 \Omega \\\\
        33 V &= 0,50 H \cdot \frac{\Delta l}{\Delta t} + 30 \Omega \\\\
        0,50 H \cdot \frac{\Delta l}{\Delta t} &= 33 V - 30 \Omega \\\\
        0,50 H \cdot \frac{\Delta l}{\Delta t} &= 3,3 V \\\\
        \frac{\Delta l}{\Delta t} &= \frac{3,3 V}{0,50 H} \\\\
        \frac{\Delta l}{\Delta t} &= 6,6 A/s \\\\
        \frac{\Delta l}{\Delta t} &= \underline{6,0 A/s}
    \end{align*}

    a) Koska $\Delta l$ / $\Delta t$ on positiivinen, virta kasvaa. \quad
    b) 6,0 A/s.
}

\end{document}
