\documentclass{report}

\input{../preamble}
\input{../macros}
\input{../letterfonts}

\title{\Huge{Anturifysiikka} Tehtävät 7}
\author{\huge{Aman Mughal}}
\date{18/03/2023}

\begin{document}

\maketitle

\qs{}{
    Käämi, jonka kierrosluku on 450 ja poikkileikkauksen halkaisija 74 mm,
    asetetaan homogeeniseen 0,23 T:n magneettikenttään. Määritä käämin
    läpäisevä magneettivuo, kun a) käämin akseli on kentän suuntainen, b) käämin
    akseli on kohtisuorassa kenttää vastaan, c) käämin akselin ja kentän välinen
    kulma on $24^\circ$. [ a) 0,45 Wb b) 0 Wb c) 0,41 Wb ]
}

\pf{Vastaus}{
    \begin{align*}
        N &= 450 \quad B = 0.23 T \quad A =  \pi \cdot \frac{ (0.074 m)^2}{4} \\\\
        a) \\
        \Phi &=  N \cdot B \cdot A \cdot \theta
        \\ &= 450 \cdot  0.23 T \cdot ( \pi \cdot \frac{ (0.074 m)^2}{4}) \cdot \cos 0^\circ
        \\ &= \underline{\underline{0.45 Wb}} \\\\
        b) \\
        \Phi &=  N \cdot B \cdot A \cdot \theta
        \\ &= 450 \cdot  0.23 T \cdot ( \pi \cdot \frac{ (0.074 m)^2}{4}) \cdot \cos 90^\circ
        \\ &= \underline{\underline{0 Wb}} \\\\
        c) \\
        \Phi &=  N \cdot B \cdot A \cdot \theta
        \\ &= 450 \cdot  0.23 T \cdot ( \pi \cdot \frac{ (0.074 m)^2}{4}) \cdot \cos 24^\circ
        \\ &= \underline{\underline{0.41 Wb}}
    \end{align*}
}

\newpage

\qs{}{
    Käämi (kierrosluku 720 ja poikkileikkauksen pinta-ala 17 $cm^2$
    ) asetetaan
    homogeeniseen magneettikenttään ja kytketään navoistaan vastukseen, jonka
    resistanssi on 44 $\Omega$. Määritä käämin induktiojännite ja piirissä kulkeva virta,
    kun magneettivuon tiheys kasvaa 1,9 sekunnissa tasaisesti arvosta 3,6 mT
    arvoon 9,9 mT. Tarkastele kahta tapausta: a) käämin akseli on
    magneettikentän suuntainen ja b) käämin akselin ja kentän välinen kulma on
    $57^\circ$. [ a) 4,1 mV ja 92 µA b) 2,2 mV ja 50 µA ]
}

\pf{Vastaus}{
    a)
    \begin{align*}
        \varepsilon &= N \frac{\Delta \phi}{\Delta t} \\
        &= N A \frac{\Delta B}{\Delta t} \\
        &= (720 \cdot 17 \cdot 10^{-4} m^2) (3.32 \cdot 10^{-3}) \\
        &= 4.06 \cdot 10^{-3} \mathrm{V}  \approx \underline{\underline{4.1 mV}} \\
        I &= \frac{\varepsilon}{R} = \frac{4.06 \cdot 10^{-3}}{44} = 92 \cdot 10^{-6}  \mathrm{A} \approx \underline{\underline{92 \mu A}}
    \end{align*}

    b)
    \begin{align*}
        \varepsilon &= N A \frac{\Delta B \cos \theta}{\Delta t} \\
        &= (720 \cdot 17 \cdot 10^{-4}) (3.32 \cdot 10^{-3}) \cos 57^\circ \\
        &= 2.21 \cdot 10^{-3} \mathrm{V} \approx \underline{\underline{2.2 mV}} \\
        I &= \frac{\varepsilon}{R} = \frac{2.21 \cdot 10^{-3}}{44} = 50 \cdot 10^{-6}  \mathrm{A} \approx \underline{\underline{50 \mu A}}
    \end{align*}
}

\newpage

\qs{}{
    Käämiä (kierrosluku 250 ja poikkileikkauksen pinta-ala 13 $cm^2$
    ) käännetään
    siten, että sen taso on aluksi vaakasuorassa ja 0,050 sekunnin kuluttua
    pystysuorassa. Määritä käämin keskimääräinen induktiojännite, kun kyseisessä
    paikassa Maan magneettikenttä on lähes pystysuora ja magneettivuon tiheys
    on 60 µT. [ 0,39 mV ]
}

\pf{Vastaus}{
    \begin{align*}
        B &= 60 \mu T \quad A = 13 cm^2 \quad N = 250 \quad \Delta t = 0.050 s \\\\
        \varepsilon &= N A \frac{\Delta B}{\Delta t} \\
        &= 250 \cdot 13 \cdot 10^{-4} m^2 \cdot \frac{60 \cdot 10^{-6} T}{0.050 s} \\
        &= 0.00039 \mathrm{V} \approx \underline{\underline{0.39 mV}}
    \end{align*}
}

\newpage

\qs{}{
    Alla olevan kuvan esittämässä systeemissä 0,25 kg:n kuorman halutaan
    laskeutuvan vakionopeudella 5,0 m/s. Vaakasuorilla kuparikiskoilla (lähes)
    kitkatta liukuva johdesauva on yhdistetty kuormaan langalla, joka kulkee
    herkkäliikkeisen pyörän yli. Homogeeninen 1,5 T:n magneettikenttä on
    pystysuora ja kiskojen välinen etäisyys 1,5 m. Kuinka suureksi on
    säätövastuksen resistanssi valittava? [ 10 Ω ] \\
    \includegraphics[width=0.5\textwidth]{./fy_teht_7_pic.png}
}

\pf{Vastaus}{
    \begin{align*}
        m &= 0,25\text{ kg}\\
        v &= 5,0\text{ m/s}\\
        g &= 9,81\text{ m/s}^2\\
        B &= 1,5\text{ T}\\
        L &= 1,5\text{ m}\\
        F_B &= m \cdot g = 0,25\text{ kg} \cdot 9,81\text{ m/s}^2 = 2,4525\text{ N}\\
        P &= F_B \cdot v = 2,4525\text{ N} \cdot 5,0\text{ m/s} = 12,2625\text{ W}\\
        U &= B \cdot L \cdot v = 1.5 \cdot 1.5 \cdot 5.0 = 11.25\ \text{V}\\
        R &= \frac{U^2}{P} = \frac{11.25^2}{12.26} = 10.32 \Omega \approx \underline{\underline{10 \Omega}}
        \end{align*}
}


\end{document}
