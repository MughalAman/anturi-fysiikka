\documentclass{report}

\input{../preamble}
\input{../macros}
\input{../letterfonts}

\title{\Huge{Anturifysiikka} Tehtävät 4}
\author{\huge{Aman Mughal}}
\date{06/02/2023}

\begin{document}

\maketitle

\qs{}{
    Ilmatäytteisen solenoidin pituus on 40 cm, ympyrän muotoisen
    poikkileikkauksen halkaisija 28 mm ja johdinkierrosten lukumäärä 900. Kuinka
    suuri virta solenoidin tulee kestää, jos magneettivuon tiheyden solenoidin
    keskiosassa halutaan olevan 35 mT? [ 12 A ]
}

\pf{Vastaus}{
    \begin{align*}
        l &= 40\ \mathrm{cm}, d = 28\ \mathrm{mm}, n = 900, B = 35\ \mathrm{mT} \\
        B &= \frac{\mu_0 n I}{l} \Rightarrow I = \frac{B \cdot l}{\mu_0 \cdot n} \\
        &= \frac{35 \cdot 10^{-3}\ \mathrm{T} \cdot 40 \cdot 10^{-2}\ \mathrm{m}}{4 \pi \cdot 10^{-7} \cdot 900} \\
        &\approx \underline{12\ \mathrm{A}}
    \end{align*}
}

\qs{}{
    Ympyrän muotoinen johdinsilmukka, jonka halkaisija on 9,1 cm, asetetaan
    homogeeniseen 0,48 T:n magneettikenttään siten, että silmukan tason
    normaalin ja kentän välinen kulma on $59^\circ$. Kuinka suuri vääntömomentti
    silmukkaan kohdistuu, kun siihen syötetään 7,4 A:n virta? [ 0,020 Nm ]
}

\pf{Vastaus}{
    \begin{align*}
        d &= 9,1\ \mathrm{cm}, B = 0,48\ \mathrm{T}, \theta = 59^\circ, I = 7,4\ \mathrm{A} \\\\
        A &= \pi \cdot \left(\frac{d}{2}\right)^2 \\
        &= \pi \cdot \left(\frac{9,1 \cdot 10^{-2}\ \mathrm{m}}{2}\right)^2 \\
        &= 0.006504\ \mathrm{m}^2 \\\\
        \theta &= 59^\circ = \frac{59 \cdot \pi}{180} \\
        &= 1,033\ \mathrm{rad} \\\\
        \tau &= B \cdot I \cdot A \cdot sin(\theta) \\
        &= 0,48\ \mathrm{T} \cdot 7,4\ \mathrm{A} \cdot 0,006504\ \mathrm{m}^2 \cdot sin(1,033\ \mathrm{rad}) \\
        &\approx \underline{0,020\ \mathrm{Nm}}
    \end{align*}
}

\newpage

\qs{}{
    Homogeeniseen 0,91 T:n magneettikenttään asetetaan käämi, jonka
    kierrosluku on 100, ympyrän muotoisen poikkileikkauksen halkaisija 7,9 cm ja
    resistanssi 5,2 $\Omega$. Käämin symmetria-akselin ja kentän välinen kulma on $71^\circ$.
    Kuinka suuri on käämin napajännitteen oltava, jotta käämiin kohdistuisi 0,021
    Nm:n vääntömomentti? [ 0,26 V ]
}

\pf{Vastaus}{
    \begin{align*}
        B &= 0,91\ \mathrm{T}, n = 100, d = 7,9\ \mathrm{cm}, R = 5,2\ \Omega, \theta = 71^\circ, \tau = 0,021\ \mathrm{Nm} \\\\
        A &= \pi \cdot \left(\frac{d}{2}\right)^2 \\
        &= \pi \cdot \left(\frac{9,1 \cdot 10^{-2}\ \mathrm{m}}{2}\right)^2 \\
        &= 0,004902\ \mathrm{m}^2 \\\\
        \theta &= 71^\circ = \frac{71 \cdot \pi}{180} \\
        &= 1,239\ \mathrm{rad} \\\\
        \tau &= n \cdot B \cdot (\frac{V}{R}) \cdot A \cdot sin(\theta) \Rightarrow V = \frac{(\tau \cdot R)}{n \cdot B \cdot A \cdot sin(\theta)} \\
        V &= \frac{(0,021\ \mathrm{Nm} \cdot 5,2\ \Omega)}{100 \cdot 0,91\ \mathrm{T} \cdot 0,004902\ \mathrm{m}^2 \cdot sin(1,239\ \mathrm{rad})} \\\
        &\approx \underline{0,26\ \mathrm{V}}
    \end{align*}
}

\qs{}{
    Tasavirtamoottorilla nostetaan vakionopeudella kuormaa, jonka massa on
    0,050 kg (ks. alla oleva kuva). Moottorin käämissä on 100 kierrosta, sen
    poikkileikkauksen pinta-ala on 12 $cm^2$
    , ja käämi on 0,10 T:n suuruisessa
    magneettikentässä. Moottorin akselin säde on 4,0 mm. Miten suuri käämin läpi
    kulkevan sähkövirran on vähintään oltava? [ 0,16 A ]
}

\pf{Vastaus}{
    \begin{align*}
        m &= 0,050\ \mathrm{kg}, n = 100, A = 12\ \mathrm{cm}^2, B = 0,10\ \mathrm{T}, r = 4\ \mathrm{mm} \\\\
        \tau &= m \cdot g \cdot r \\
        &= 0,050\ \mathrm{kg} \cdot 9,81\ \mathrm{m/s}^2 \cdot 4 \cdot 10^{-3}\ \mathrm{m} \\\\
        I &= \frac{\tau}{B \cdot A} \\
        &= \frac{(0,050\ \mathrm{kg} \cdot 9,81\ \mathrm{m/s}^2 \cdot 4 \cdot 10^{-3}\ \mathrm{m})}{100 \cdot 0,10\ \mathrm{T} \cdot 12 \cdot 10^{-4}\ \mathrm{m}^2} \\
        &\approx \underline{0,16\ \mathrm{A}}
    \end{align*}
}

\end{document}
